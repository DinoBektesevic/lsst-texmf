% Copyright 2006 by Till Tantau
%
% This file may be distributed and/or modified
%
% 1. under the LaTeX Project Public License and/or
% 2. under the GNU Free Documentation License.
%
% See the file doc/generic/pgf/licenses/LICENSE for more details.


\section{Fading Library}
\label{section-library-fadings}

\begin{pgflibrary}{fadings}
  The package defines a number of fadings, see
  Section~\ref{section-tikz-transparency} for an introduction.  The
  \tikzname\ version defines special \tikzname\ commands for creating
  fadings. These commands are explained in
  Section~\ref{section-tikz-transparency}.   
\end{pgflibrary}

\newcommand\fadingindex[1]{%
  \index{#1@\protect\texttt{#1} fading}%
  \index{Fadings!#1@\protect\texttt{#1}}%
  \texttt{#1}& 
  \begin{tikzpicture}[baseline=5mm-.5ex]
    \fill [black!20] (0,0) rectangle (1,1);
    \path [pattern=checkerboard,pattern color=black!30] (0,0) rectangle (1,1);

    \fill [path fading=#1,blue] (0,0) rectangle (1,1);
  \end{tikzpicture} \\[4.5mm]
}

\noindent
\begin{tabular}{ll}
  \emph{Fading name} & \emph{Example (solid blue faded on checkerboard)} \\[1mm]
  \fadingindex{west}  
  \fadingindex{east}  
  \fadingindex{north}  
  \fadingindex{south} 
  \fadingindex{circle with fuzzy edge 10 percent} 
  \fadingindex{circle with fuzzy edge 15 percent} 
  \fadingindex{circle with fuzzy edge 20 percent} 
  \fadingindex{fuzzy ring 15 percent} 
\end{tabular}


%%% Local Variables: 
%%% mode: latex
%%% TeX-master: "pgfmanual-pdftex-version"
%%% End: 
