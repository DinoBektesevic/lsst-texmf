\documentclass[DM,lsstdraft,authoryear,toc]{lsstdoc}

% Package imports go here

% Local commands go here

\title[Short title]{Title of document}

\author{
A.~Author,
B.~Author,
and
C.~Author}

\setDocRef{DMTN-nnn}
\date{\today}

\setDocAbstract{%
This document demonstrates how to use the LSST \LaTeX\ class files to make a Data Management
tech note. Build this document in the normal way, making sure that the class file is
available in the \LaTeX\ load path.
}

% Change history defined here. Will be inserted into
% correct place with \maketitle
% OLDEST FIRST: VERSION, DATE, DESCRIPTION, OWNER NAME
\setDocChangeRecord{%
\addtohist{1}{2017-04-17}{Initial release. Based on LDM example}{Tim Jenness}
\addtohist{2}{yyyy-mm-dd}{Future changes}{Future person}
}

\begin{document}

% Create the title page
% Table of contents will be added automatically if "toc" class option
% is used.
\maketitle

\section{Introduction}

Now write your document as you would normally write it.
Different citation schemes are supported, and the default bibliography style is declared by the class.
In this example we have enabled author-year citing.
Use \verb|\citeds| for citing docushare documents.

\verb|\citedsp|: \citedsp{LPM-17, LSE-30} \\
\verb|\citeds|: (SRD; \citeds{LPM-17,LSE-29}) \\
\verb|\citep[][]|: \citep[e.g.,][are interesting]{2009arXiv0912.0201L,2016SPIE.9913E..0GJ} \\
\verb|\cite|: \cite{LPM-51,Wang:2011:QDS:2063348.2063364}

Font checking: \texttt{Fixed width font}, \textsc{Small Caps}, \textbf{Bold}, \textit{Italic}, \textbf{\textit{BoldItalic}}.

Math checking: $A = \pi r^2 \smalltilde\mathrm{(math roman)}\mathit{(italic)}\mathsf{(sans serif)} 0 == \zerob \xib$

\begin{equation}
O(x, z) = \sum_\lambda I(x',\lambda) \otimes D(\lambda, z)
\end{equation}

Talk about something that relates to a requirement.\ossreq{1234}

\newtext{We can show new text} \oldtext{and text to be removed}.

\begin{note}[Note Title]
  This is used for note blocks.
\end{note}

% Include all the relevant bib files.
% lsst is for DocuShare and DMTN entries.
% refs_ads is for entries coming from ADS.
\bibliography{lsst,lsst-dm,refs,books,refs_ads}

\end{document}
