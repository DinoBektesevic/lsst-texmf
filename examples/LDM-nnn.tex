\documentclass[DM,lsstdraft]{lsstdoc}

\usepackage[english]{babel}
\usepackage[utf8x]{inputenc}
\usepackage{amsmath}
\usepackage{graphicx}
\usepackage{longtable}
\usepackage{hyperref}
\usepackage{comment}
\usepackage{natbib}

\excludecomment{changelog}
\excludecomment{todo}
\excludecomment{openissues}

% Local commands go here
\newcommand{\G}[1]{{\color{green} #1}}
\newcommand{\B}[1]{{\color{blue} #1}}
\newcommand{\R}[1]{{\color{red} #1}}

%% Journal abbreviations
\bibliographystyle{aasjournal}

\title{Title of document}

\author{
A.~Author,
B.~Author,
and
C.~Author}

\setDocRef{LDM-nnn}
\setDocDate{\today}
\setDocRevision{TBD}
\setDocStatus{draft}
\setDocTitle[Short title]{Unused main title}  % Use this to set the short title for now

\setDocAbstract{%
This document demonstrates how to use the LSST \LaTeX\ class files to make Data Management
documents. Build this document in the normal way, making sure that the class file is
available in the \LaTeX\ load path.
}

\begin{document}

% Create the title page
\maketitle

% Change history goes after the title page
% MOST RECENT FIRST
\begin{docHistory}
  \addtohist{2}{1}{YYYY-MM-DD}{??}{Future changes}
  \addtohist{2}{0}{2017-09-10}{TJ}{Initial release based on Gaia examples.}
\end{docHistory}

% Create the table of contents
\clearpage
\tableofcontents
\clearpage

\section{Introduction}

Now write your document as you would normally write it.

\end{document}
