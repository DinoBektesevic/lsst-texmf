
\documentclass[{{ cookiecutter.org }},lsstdraft,toc]{lsstdoc}

{# skip draft watermark for technotes and use author-year citations #}
\documentclass[{{ cookiecutter.org }},authoryear,toc]{lsstdoc}

% lsstdoc documentation: https://lsst-texmf.lsst.io/lsstdoc.html

% Package imports go here.

% Local commands go here.
\input{aglossary.tex}
\makeglossaries


% To add a short-form title:
% \title[Short title]{Title}
\title{ {{- cookiecutter.title -}} }

% Optional subtitle
% \setDocSubtitle{A subtitle}


\author{%
{{ cookiecutter.first_author }}
}

\setDocRef{ {{- cookiecutter.series.upper() -}}-{{- cookiecutter.serial_number -}} }

\date{\today}

% Optional: name of the document's curator
% \setDocCurator{The Curator of this Document}


\setDocAbstract{%
{{ cookiecutter.abstract }}
}

% Change history defined here.
% Order: oldest first.
% Fields: VERSION, DATE, DESCRIPTION, OWNER NAME.
% See LPM-51 for version number policy.

\setDocChangeRecord{%
  \addtohist{1}{YYYY-MM-DD}{Unreleased.}{ {{- cookiecutter.first_author -}} }
}

\begin{document}

% Create the title page.
% Table of contents is added automatically with the "toc" class option.

\maketitle

\mkshorttitle
%switch to \maketitle if you wan the title page and toc



% ADD CONTENT HERE ... a file per section can be good for editing
\section{Introduction} \label{sec:intro}
This is in the file body.tex. Put your text here.

The appendices for the \appref{sec:bib} and \appref{sec:acronyms} are defined in the main file CODE-nnn.tex.

The main bibliography file is in the lsst-texmf/texmf/bibtex/bib so you can refer to documents such as \citeds{LDM-294} (from lsst.bib)  or papers like \cite{2008arXiv0805.2366I} (from refs\_ads.bib). You may make a PR to add new refs to these files.

Acronyms like BOE and BAC will be picked up by generateAcronyms.py which is in lsst-texmf/bin -- that needs to be in the PATH.


\appendix
% Include all the relevant bib files.
% https://lsst-texmf.lsst.io/lsstdoc.html#bibliographies
\section{References} \label{sec:bib}
\bibliography{lsst,lsst-dm,refs_ads,refs,books}

%Make sure lsst-texmf/bin/generateAcronyms.py is in your path
%\section{Acronyms used in this document}\label{sec:acronyms}
%The following table has been generated from the on-line Gaia acronym list:
\newline\newline%decrement table counter so table sin doc start at 1.
\addtocounter{table}{-1}
\begin{longtable}{|l|p{0.8\textwidth}|}\hline 
\textbf{Acronym} & \textbf{Description}  \\\hline
ACP&Astro Coarse Phase \\\hline
CU&Coordination Unit (in DPAC) \\\hline
DM&Data Model \\\hline
INT&INTterferogram (Code V format) \\\hline
LSST&Large-aperture Synoptic Survey Telescope \\\hline
PPN&Parametrised Post-Newtonian (formalism in gravitational physics) \\\hline
SRS&Software Requirements Specification \\\hline
STS&Star Tracker System \\\hline
SVTP&Software Verification Test Plan \\\hline
\end{longtable} 

%Unccomment these if you want old style acronyms

\input{aglossary.tex}
\end{document}
